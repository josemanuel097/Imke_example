% Generated by GrindEQ Word-to-LaTeX 
\documentclass{article} %%% use \documentstyle for old LaTeX compilers

\usepackage[english]{babel} %%% 'french', 'german', 'spanish', 'danish', etc.
\usepackage{amssymb}
\usepackage{amsmath}
\usepackage{txfonts}
\usepackage{mathdots}
\usepackage[classicReIm]{kpfonts}
\usepackage[dvips]{graphicx} %%% use 'pdftex' instead of 'dvips' for PDF output
\usepackage[left]{lineno}
\linenumbers
% You can include more LaTeX packages here 


\begin{document}



\noindent \textit{\includegraphics*[width=1.84in, height=0.47in, keepaspectratio=false]{image1}\includegraphics*[width=0.59in, height=0.39in, keepaspectratio=false]{image2}}

\noindent \textit{Cancers }\textbf{2020}, \textit{12}, x; doi: FOR PEER REVIEW www.mdpi.com/journal/cancers

\noindent \textit{Type of the Paper (Article, Review, Communication, etc.)}

\noindent \textbf{Title}

\noindent \textbf{Firstname Lastname ${}^{1}$, Firstname Lastname ${}^{2}$ and Firstname Lastname ${}^{2,}$*}

\noindent ${}^{1}$ Affiliation 1; e-mail@e-mail.com

\noindent ${}^{2}$ Affiliation 2; e-mail@e-mail.com

\noindent \textbf{*} Correspondence: e-mail@e-mail.com; Tel.: (optional; include country code; if there are multiple corresponding authors, add author initials) +xx-xxxx-xxx-xxxx (F.L.)



\noindent 

\noindent \textbf{0. How to Use This Template}

The template details the sections that can be used in a manuscript. Note that each section has a corresponding style, which can be found in the `Styles' menu of Word. Sections that are not mandatory are listed as such. The section titles given are for Articles. Review papers and other article types have a more flexible structure. 

Remove this paragraph and start section numbering with 1. For any questions, please contact the editorial office of the journal or support@mdpi.com.

\noindent \textbf{1. Introduction}

The introduction should briefly place the study in a broad context and highlight why it is important. It should define the purpose of the work and its significance. The current state of the research field should be reviewed carefully and key publications cited. Please highlight controversial and diverging hypotheses when necessary. Finally, briefly mention the main aim of the work and highlight the principal conclusions. As far as possible, please keep the introduction comprehensible to scientists outside your particular field of research. References should be numbered in order of appearance and indicated by a numeral or numerals in square brackets, e.g., [1] or [2,3], or [4--6]. See the end of the document for further details on references.

\noindent \textbf{2. Results}

This section may be divided by subheadings. It should provide a concise and precise description of the experimental results, their interpretation as well as the experimental conclusions that can be drawn.

\noindent \textit{\eject 2.1. Subsection}

\noindent 2.1.1. Subsubsection

Bulleted lists look like this:

\begin{enumerate}
\item  First bullet

\item  Second bullet

\item  Third bullet
\end{enumerate}

Numbered lists can be added as follows:

\begin{enumerate}
\item  First item

\item  Second item

\item  Third item
\end{enumerate}

The text continues here.

\noindent \textit{2.2. Figures, Tables and Schemes}

All figures and tables should be cited in the main text as Figure 1, Table 1, etc.

\begin{tabular}{|p{2.1in}|p{2.1in}|} \hline 
\includegraphics*[width=1.37in, height=1.37in, keepaspectratio=false]{image3}\newline (\textbf{a}) & \includegraphics*[width=1.37in, height=1.37in, keepaspectratio=false]{image4}\newline (\textbf{b}) \\ \hline 
\end{tabular}

\textbf{Figure 1.} This is a figure, Schemes follow the same formatting. If there are multiple panels, they should be listed as: (\textbf{a}) Description of what is contained in the first panel; (\textbf{b}) Description of what is contained in the second panel. Figures should be placed in the main text near to the first time they are cited. A caption on a single line should be centered.

\noindent \textbf{Table 1.} This is a table. Tables should be placed in the main text near to the first time they are cited.

\begin{tabular}{|p{0.7in}|p{0.7in}|p{0.7in}|} \hline 
\textbf{Title 1} & \textbf{Title 2} & \textbf{Title 3} \\ \hline 
entry 1 & data & data \\ \hline 
entry 2 & data & data ${}^{1}$ \\ \hline 
\end{tabular}

${}^{1}$ Tables may have a footer.

\noindent \textit{2.3. Formatting of Mathematical Components}

This is an example of an equation:

\begin{tabular}{|p{3.9in}|p{0.2in}|} \hline 
a = 1, & \eqref{GrindEQ__1_} \\ \hline 
\end{tabular}

the text following an equation need not be a new paragraph. Please punctuate equations as regular text.

Theorem-type environments (including propositions, lemmas, corollaries etc.) can be formatted as follows:

\noindent \textbf{Theorem 1.}\textit{ Example text of a theorem. Theorems, propositions, lemmas, }etc.\textit{ should be numbered sequentially (i.e., Proposition 2 follows Theorem 1). Examples or Remarks use the same formatting, but should be numbered separately, so a document may contain Theorem 1, Remark 1 and Example 1.}

The text continues here. Proofs must be formatted as follows:

\noindent \textbf{Proof of Theorem 1.} Text of the proof. Note that the phrase `of Theorem 1' is optional if it is clear which theorem is being referred to. Always finish a proof with the following symbol. ?

The text continues here.

\noindent \textbf{3. Discussion}

Authors should discuss the results and how they can be interpreted in perspective of previous studies and of the working hypotheses. The findings and their implications should be discussed in the broadest context possible. Future research directions may also be highlighted.

\noindent \textbf{4. Materials and Methods }

Materials and Methods should be described with sufficient details to allow others to replicate and build on published results. Please note that publication of your manuscript implicates that you must make all materials, data, computer code, and protocols associated with the publication available to readers. Please disclose at the submission stage any restrictions on the availability of materials or information. New methods and protocols should be described in detail while well-established methods can be briefly described and appropriately cited.

Research manuscripts reporting large datasets that are deposited in a publicly available database should specify where the data have been deposited and provide the relevant accession numbers. If the accession numbers have not yet been obtained at the time of submission, please state that they will be provided during review. They must be provided prior to publication.

Interventionary studies involving animals or humans, and other studies require ethical approval must list the authority that provided approval and the corresponding ethical approval code. 

\noindent \textbf{5. Conclusions}

This section is mandatory, with one or two paragraphs.

\noindent \textbf{6. Patents}

This section is not mandatory, but may be added if there are patents resulting from the work reported in this manuscript.

\noindent \textbf{Supplementary Materials:} The following are available online at www.mdpi.com/xxx/s1, Figure S1: title, Table S1: title, Video S1: title. 

\noindent \textbf{Author Contributions: }For research articles with several authors, a short paragraph specifying their individual contributions must be provided. The following statements should be used ``conceptualization, X.X. and Y.Y.; methodology, X.X.; software, X.X.; validation, X.X., Y.Y. and Z.Z.; formal analysis, X.X.; investigation, X.X.; resources, X.X.; data curation, X.X.; writing---original draft preparation, X.X.; writing---review and editing, X.X.; visualization, X.X.; supervision, X.X.; project administration, X.X.; funding acquisition, Y.Y.'', please turn to the CRediT taxonomy for the term explanation. Authorship must be limited to those who have contributed substantially to the work reported.

\noindent \textbf{Funding: }Please add: ``This research received no external funding'' or ``This research was funded by NAME OF FUNDER, grant number XXX'' and ``The APC was funded by XXX''. Check carefully that the details given are accurate and use the standard spelling of funding agency names at https://search.crossref.org/funding, any errors may affect your future funding.

\noindent \textbf{Acknowledgments:} In this section you can acknowledge any support given which is not covered by the author contribution or funding sections. This may include administrative and technical support, or donations in kind (e.g., materials used for experiments).

\noindent \textbf{Conflicts of Interest:} Declare conflicts of interest or state ``The authors declare no conflict of interest.'' Authors must identify and declare any personal circumstances or interest that may be perceived as inappropriately influencing the representation or interpretation of reported research results. Any role of the funders in the design of the study; in the collection, analyses or interpretation of data; in the writing of the manuscript, or in the decision to publish the results must be declared in this section. If there is no role, please state ``The funders had no role in the design of the study; in the collection, analyses, or interpretation of data; in the writing of the manuscript, or in the decision to publish the results''.

\noindent \textbf{Appendix A}

The appendix is an optional section that can contain details and data supplemental to the main text. For example, explanations of experimental details that would disrupt the flow of the main text, but nonetheless remain crucial to understanding and reproducing the research shown; figures of replicates for experiments of which representative data is shown in the main text can be added here if brief, or as Supplementary data. Mathematical proofs of results not central to the paper can be added as an appendix.

\noindent \textbf{Appendix B}

All appendix sections must be cited in the main text. In the appendixes, Figures, Tables, etc. should be labeled starting with `A', e.g., Figure A1, Figure A2, etc. 

\noindent \textbf{References}

\noindent References must be numbered in order of appearance in the text (including citations in tables and legends) and listed individually at the end of the manuscript. We recommend preparing the references with a bibliography software package, such as EndNote, ReferenceManager or Zotero to avoid typing mistakes and duplicated references. Include the digital object identifier (DOI) for all references where available.

\noindent 

\noindent Citations and References in Supplementary files are permitted provided that they also appear in the reference list here. 

\noindent 

\noindent In the text, reference numbers should be placed in square brackets [ ], and placed before the punctuation; for example [1], [1--3] or [1,3]. For embedded citations in the text with pagination, use both parentheses and brackets to indicate the reference number and page numbers; for example [5] (p. 10), or [6] (pp. 101--105).

\noindent 

\begin{enumerate}
\item  Author 1, A.B.; Author 2, C.D. Title of the article. \textit{Abbreviated Journal Name} \textbf{Year}, \textit{Volume}, page range.

\item  Author 1, A.; Author 2, B. Title of the chapter. In \textit{Book Title}, 2nd ed.; Editor 1, A., Editor 2, B., Eds.; Publisher: Publisher Location, Country, 2007; Volume 3, pp. 154--196.

\item  Author 1, A.; Author 2, B. \textit{Book Title}, 3rd ed.; Publisher: Publisher Location, Country, 2008; pp. 154--196.

\item  Author 1, A.B.; Author 2, C. Title of Unpublished Work. \textit{Abbreviated Journal Name} stage of publication (under review; accepted; in press).

\item  Author 1, A.B. (University, City, State, Country); Author 2, C. (Institute, City, State, Country). Personal communication, 2012.

\item  Author 1, A.B.; Author 2, C.D.; Author 3, E.F. Title of Presentation. In Title of the Collected Work (if available), Proceedings of the Name of the Conference, Location of Conference, Country, Date of Conference; Editor 1, Editor 2, Eds. (if available); Publisher: City, Country, Year (if available); Abstract Number (optional), Pagination (optional).

\item  Author 1, A.B. Title of Thesis. Level of Thesis, Degree-Granting University, Location of University, Date of Completion.

\item  Title of Site. Available online: URL (accessed on Day Month Year).
\end{enumerate}

\begin{tabular}{|p{0.8in}|p{3.3in}|} \hline 
\includegraphics*[width=1.10in, height=0.40in, keepaspectratio=false]{image5} & {\copyright} 2020 by the authors. Submitted for possible open access publication under the terms and conditions of the Creative Commons Attribution (CC BY) license (http://creativecommons.org/licenses/by/4.0/). \\ \hline 
\end{tabular}



\noindent 

\noindent 

\noindent For all Western blot figures, please include densitometry readings/intensity ratio of each band. In addition, please include the whole blot (uncropped blots) showing all the bands with all molecular weight markers on the Western in the Supplemental Materials

\noindent E.g.: 

\noindent \includegraphics*[width=6.13in, height=3.56in, keepaspectratio=false]{image6}


\end{document}

